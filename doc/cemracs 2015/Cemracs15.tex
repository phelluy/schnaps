\documentclass[a4paper,twoside,french,english,11pt]{article}
\usepackage{amsmath}
\usepackage{amsthm}
\usepackage{lastpage}
\usepackage{fancyhdr}
%\usepackage{a4wide}
\usepackage[a4paper,hcentering,vcentering,bindingoffset=0mm]{geometry}
\usepackage{txfonts}
\usepackage[T1]{fontenc}
\usepackage[latin9]{inputenc}
%\geometry{verbose,a4paper,tmargin=3cm,bmargin=3cm,lmargin=3cm,rmargin=3cm,headheight=3cm,headsep=3cm}
\usepackage{babel}
\usepackage[dvipsnames,svgnames,x11names,hyperref]{xcolor}
%\usepackage{xcolor}
\usepackage{graphicx}
%\usepackage{graphics}
\usepackage{epsfig}
%\usepackage{esint}
%\usepackage{color}
\usepackage{amsfonts}
\usepackage{amssymb,latexsym}
\usepackage{amscd}
\usepackage{multirow}
\usepackage{graphicx}
\usepackage{amssymb}
\usepackage{bm}
\usepackage{cancel}
\usepackage{multirow}
\usepackage{lineno}
\usepackage{setspace}
\usepackage{enumitem}
\usepackage{booktabs}
\usepackage{multirow}
\usepackage{stmaryrd}
\usepackage{tikz}
\usepackage{pifont}% http://ctan.org/pkg/pifont

\usepackage[pdftex, bookmarks=true, bookmarksopen=true, bookmarksnumbered
,bookmarksopenlevel=1
,colorlinks=true, linkbordercolor=white, citecolor=DarkBlue, linkcolor=DarkBlue
,pdftitle={Hierarchy of fluids models for magnetized and collisional plasmas}]{hyperref}

\usepackage{comment} 


\newcommand{\cmark}{\ding{51}}%
\newcommand{\xmark}{\ding{55}}%
\newcommand{\mmark}{\ding{169}}%
\newcommand{\dr}{\partial_R}
\newcommand{\dy}{\partial_y}
\newcommand{\dx}{\partial_x}
\newcommand{\ds}{\displaystyle}
\newcommand{\dz}{\partial_Z}
\newcommand{\dphi}{\partial_{\phi}}
\newcommand{\ephi}{\bm{e}_{\phi}}
\newcommand{\dt}{\partial_t}
\newcommand{\lp}{\Delta_{pol}}
\newcommand{\lpp}{\Delta_{pol}^2}
\newcommand{\gradp}{\nabla_{pol}}
\newcommand{\lpm}{\Delta_{pol}^{-1}}
\newcommand{\gs}{\Delta^{*}}
\newcommand{\U}{\bm{U}}
\newcommand{\V}{\bm{V}}
\newcommand{\vv}{\bm{v}}
\newcommand{\vW}{\bm{W}}
\newcommand{\vB}{\bm{B}}
\newcommand{\vJ}{\bm{J}}
\newcommand{\vE}{\bm{E}}
\newcommand{\rot}{\nabla \times}
\newcommand{\er}{\bm{e}_R}
\newcommand{\ez}{\bm{e}_Z}
\newcommand{\f}{f(\x,\vw,t)}
\newcommand{\g}{g(\x,\vw,t)}
\newcommand{\eps}{\varepsilon}
\newcommand{\intv}{\int_{\mathbb{R}^3}}
\newcommand{\vw}{\bm{v}}
\newcommand{\vdelta}{\bm{k}}
\newcommand{\x}{\bm{x}}
\newcommand{\intx}{\int_{D_{\bm{x}}}}
\newcommand{\diffcolor}{\textcolor{blue}}
\newcommand{\twofluidcolor}{\textcolor{magenta}}
\newcommand{\termsourcecolor}{\textcolor{darkgreen}}
\newcommand{\st}{\overline{\overline{\bm{\Pi}}}}

\usetikzlibrary{backgrounds}
\usetikzlibrary{mindmap,trees}	% For mind map

\newtheorem{theorem}{Theorem}[section]
\newtheorem{lemma}[theorem]{Lemma}
\newtheorem{proposition}[theorem]{Proposition}
\newtheorem{corollary}[theorem]{Corollary}
\newtheorem{definition}{Definition}[section]
\newtheorem{assump}{Assumptions}[section]
\newtheorem{remark}[theorem]{Remark}
\newtheorem{cri}[theorem]{Criterion}
\newtheorem{On}[theorem]{Ongoing works}


\title{Physic-Based preconditioning for hyperbolic system and DG discretization}

\author{C. Courtes\footnotemark[4], E. Franck\footnotemark[1],\quad H. Guillard \footnotemark[3], \quad P. Helluy \footnotemark[3],H. Oberlin footnotemark[3]}

\begin{document}

\maketitle


\footnotetext[1]{Inria Nancy grand Est, TONUS Team, Strasbourg, France}
\footnotetext[2]{Max-Planck-Institut f\"ur Plasmaphysik, Boltzmannstra\ss e 2D-85748 Garching, Germany.}
\footnotetext[3]{Inria Sofia-Antipolis, Castor team, Nice, France .}
\footnotetext[4]{Universit\'e Paris 11. Orsay.}
\tableofcontents

\section{Model study and Discretization}
We propose to study the discretization of the problem :
$$
\left\{\begin{array}{l}
\ds \dt p+c\nabla \cdot \bm{u}=0\\
\ds \dt \bm{u}+c\nabla p = 0 \end{array}\right.
$$
Before we propose to add some admissible boundary condition. Defining $p=\dt f$ and $\bm{u}=- c\nabla f$ we obtain that $f$ is a solution of 
$$
\partial_{tt} f-c^2\Delta f=0
$$
A classical boundary condition for this problem is $\lambda \dt f+f=g$. Using the definition of $\bm{u}$ and $p$ we obtain the following hyperbolic problem
$$
\left\{\begin{array}{l}
\ds \dt p+c\nabla \cdot \bm{u}=0\\
\ds \dt \bm{u}+c\nabla p = 0 \\
\lambda p-c  (\bm{u},\bm{n})=g\quad \delta \Omega \end{array}\right.
$$
Firstly we propose a energy estimate for this problem. For this we multiply the first equation by $p$ and the second one by $\bm{u}$ and we integrate. We obtain
\begin{align*}
& \frac{1}{2}\dt\int_{\Omega}(p^2+(\bm{u},\bm{u}))+c\int_{\Omega}p\nabla \cdot \bm{u} + c\int_{\Omega} \bm{u} \nabla p\\
&=  \frac{1}{2}\dt \int_{\Omega}(p^2+(\bm{u},\bm{u}))+c\int_{\Omega}p\nabla \cdot \bm{u} - c\int_{\Omega}p\nabla \cdot \bm{u}+c\int_{\partial \Omega} p (\bm{u},\bm{n})\\
&=  \frac{1}{2}\dt \int_{\Omega}(p^2+(\bm{u},\bm{u}))+\int_{\partial \Omega} \lambda c^2 (\bm{u},\bm{n})^2+ \int_{\partial \Omega} c g (\bm{u},\bm{n})
\end{align*}
at the end we obtain
$$
\frac{1}{2}\dt \int_{\Omega}(p^2+(\bm{u},\bm{u}))=-\int_{\partial \Omega} \frac{c^2}{\lambda} (\bm{u},\bm{n})^2- \int_{\partial \Omega} \frac{c g}{\lambda} (\bm{u},\bm{n})
$$
In the homogeneous case the boundary condition is dissipative.
TOOOOODDOOO BC 
\section{Time and spatial discretization}
The classical explicit scheme for this problem are not useful is the velocity $c>>1$. Indeed the time step is constrained by this velocity. Consequently use an implicit scheme is an interesting way (there also a possibility to construct semi-implicit AP scheme with a CFL condition independent of $\eps$). Now we propose to write an implicit scheme based on a Cranck-Nicholson scheme :
$$
\left\{\begin{array}{l}
\ds p^{n+1} + \theta c\Delta t \nabla \cdot \mathbf{u}^{n+1}=  p^n- (1-\theta) c \Delta t   \nabla \cdot \mathbf{u}^{n}\\
\ds \mathbf{u}^{n+1}+\theta c \Delta t \nabla p^{n+1}= \mathbf{u}^n-(1-\theta ) c\Delta t \nabla p^{n}\end{array}\right.
$$
We add the boundary condition $ \lambda  p^{n+1}-c(\bm{u}^{n+1},\bm{n})=g$.
\subsection{Finite element method and spatial discretization}
To discretize the previous semi discrete scheme with finite element method. To begin we define three test function $v_p$, $v_{\bm{u}}=(v_{u1},v_{u2})$. Firstly we consider the weak form multiplying the problem by thee basis function. We will begin by the first equation
$$
\int_{\Omega}p^{n+1}v_p+ c\theta\Delta t  \int_{\Omega}\nabla \cdot \mathbf{u}^{n+1}v_p=  \int_{\Omega}p^n v_p- (1-\theta) c\Delta t  \int_{\Omega}\nabla \cdot \mathbf{u}^{n} v_p
$$
Integrating by part we obtain
\begin{align*}
&\int_{\Omega}p^{n+1}v_p -c\theta \Delta t  \int_{\Omega} \mathbf{u}^{n+1}\cdot \nabla v_p+ c\theta\Delta t  \int_{\partial \Omega} (\mathbf{u}^{n+1},\bm{n}) v_p\\
& =  \int_{\Omega}p^n v_p +(1-\theta) c\Delta t  \int_{\Omega}\mathbf{u}^{n} \cdot \nabla v_p -c (1-\theta)\Delta t  \int_{\partial \Omega} (\mathbf{u}^n,\bm{n}) v_p
\end{align*}
\begin{align*}
&\int_{\Omega}p^{n+1}v_p -c\theta\Delta t  \int_{\Omega} \mathbf{u}^{n+1}\cdot \nabla v_p+ c\theta\Delta t  \int_{\partial \Omega} \frac{(p^{n+1}-g) v_p}{\lambda c}\\
& =  \int_{\Omega}p^n v_p +(1-\theta) c\Delta t  \int_{\Omega}\mathbf{u}^{n} \cdot \nabla v_p -c (1-\theta)\Delta t  \int_{\partial \Omega} \frac{(p^n-g) v_p}{\lambda c}
\end{align*}
Now we introduce the weak form for the other equations.
\begin{align*}
&\int_{\Omega}\bm{u}^{n+1}v_{\bm{u}} -c\theta \Delta t \int_{\Omega} p^{n+1}\nabla \cdot v_{\bm{u}}+ c\theta\Delta t  \int_{\partial \Omega} (v_{\bm{u}},\bm{n}) (c\lambda(\bm{u}^{n+1},\bm{n}) +g)\\
& =  \int_{\Omega}\bm{u}^{n} v_{\bm{u}} +(1-\theta) c\Delta t  \int_{\Omega} p^n\nabla \cdot v_{\bm{u}} -c (1-\theta)\Delta t  \int_{\partial \Omega} (v_{\bm{u}},\bm{n}) (c\lambda(\bm{u}^{n},\bm{n}) +g)
\end{align*}
Now we define the basis function $\phi_i(\mathbf{x})$ associated at the degree of freedom $i$ (total of degree of freedom). To obtain the final system we define $v_p=v_{u1}=v_{u2}=\phi_i$ and $p^n=\sum_{j=1}^N p_j^n \phi_j(\mathbf{x})$ (same for other variables).
We apply the same procedure to the other equation we obtain the system $A_I \mathbf{U}^{n+1}=A_E\mathbf{U}^{n}$ which is given by
$$
\left(\begin{array}{lll}
M + \theta B_{u}(n_1,n_1) & \theta B_{u}(n_1,n_2) & \theta D_1 \\
\theta B_{u}(n_2,n_1)  & M +\theta B_{u}(n_2,n_2) & \theta D_2 \\
 \theta D_1 & \theta D_2  & M+ \theta B_p
\end{array}\right)\left(\begin{array}{l} u_{1,h}^{n+1} \\ u_{2,h}^{n+1}\\ p_h^{n+1}  \end{array}\right)
$$
$$
=\left(\begin{array}{lll}
M - (1- \theta )B_{u}(n_1,n_1) &  - (1- \theta ) B_{u}(n_1,n_2)  &  - (1- \theta )D_1 \\
 - (1- \theta ) B_{u}(n_2,n_1)  & M - (1- \theta ) B_{u}(n_2,n_2) &  - (1- \theta )D_2  \\
  - (1- \theta ) D_1 & - (1- \theta )D_2 & M - (1- \theta ) B_p
\end{array}\right)\left(\begin{array}{l} u_{1,h}^{n} \\ u_{2,h}^{n} \\ p_h^{n} \end{array}\right)+\left(\begin{array}{l} b_{u1} \\ b_{u2} \\b_{p} \end{array}\right)
$$
with $M$ the mass matrix, $D_{1,ij}=c \Delta t\left(\int_{\Omega} \phi_j\dx \phi_i\right)$, $D_{2,ij}=c \Delta t\left(\int_{\Omega} \phi_j\dy \phi_i\right)$, $B_{p,ij}=c\Delta t\left(\int_{\partial \Omega} \frac{\phi_i \phi_j}{\lambda c}\right)$ and $B_{u,ij}(n_1,n_2)=c\Delta t\left(\int_{\partial \Omega} n_1n_2 c\lambda\phi_i \phi_j\right)$. \\\\
The Rhs are given by $b_{p,i}=c\Delta t\left(\int_{\partial \Omega} \frac{g\phi_i }{\lambda c}\right)$, $b_{u1,i}=-c\Delta t\left(\int_{\partial \Omega} \lambda c n_1 g\phi_i \right)$ and $b_{u2,i}=-c\Delta t\left(\int_{\partial \Omega} \lambda c n_2 g\phi_i\right)$.
\section{Principle of the preconditioning for the wave equation}
Now we propose to construct a preconditioning for the wave equation.
The idea proposed here is to construct a preconditioning  using a simplified and modified (to treat the stiffness) form of the equations which is close to a diffusion equation. This parabolic modified and simplified equations which can be solved easily with multi-grid methods gives the preconditioning. We consider
\subsection{Schur on the pressure}
To begin we write the same method but we compute the schur on the pressure. We obtain the following algorithm
The system is given by 
$$
\left(\begin{array}{ll} 
D  & U \\
 L & M\end{array}\right)\left(\begin{array}{l} 
 \mathbf{u}^{n+1}\\
 p^{n+1} \end{array}\right)=\left(\begin{array}{l} 
 R_\mathbf{u}\\
 R_p \end{array}\right)
 $$
with $M=I_d$,
 $$
 D=\left(\begin{array}{ll} 
I_d  & 0 \\
 0 & I_d\end{array}\right),\quad
  L=\left(\begin{array}{ll} 
\ds\theta c\Delta t  \dx  & \ds\theta c\Delta t  \dy \end{array}\right),\quad
  U=\left(\begin{array}{l} 
\ds  \theta c \Delta t  \dx  \\
\ds \theta c \Delta t\dy \end{array}\right)
 $$
 and $\ds \mathbf{u}^{n+1}=\left(\begin{array}{ll} 
u_1^{n+1}  & u_2^{n+1} \end{array}\right)$. 
Consequently we can write the solution on the following form
$$
\left(\begin{array}{l} 
 \mathbf{u}^{n+1}
 p^{n+1} \end{array}\right) =\left(\begin{array}{ll}
D & U  \\
L & M\end{array}\right)^{-1}\left(\begin{array}{l} 
 R_\mathbf{u}\\
 R_p \end{array}\right)
$$
$$
=\left(\begin{array}{ll}
I & D^{-1} U  \\
0 & I\end{array}\right)\left(\begin{array}{ll}
D^{-1} & 0  \\
0 & P_{schur}^{-1}\end{array}\right)\left(\begin{array}{ll}
I & 0  \\
-LD^{-1} & I\end{array}\right)\left(\begin{array}{l} 
 R_\mathbf{u}\\
 R_p\end{array}\right)
 $$
with $\textcolor{red}{P_{schur}=M-LD^{-1}U}$. Solving the system with this decomposition is equivalent to solve the following algorithm 
$$
\left.\begin{array}{l}
\mbox{Predictor :\quad} D \textcolor{red}{\mathbf{u}^*}=R_{\mathbf{u}}\\
\mbox{Potential evolution :\quad} P_{schur}\textcolor{red}{p^{n+1}}=\left(-L\textcolor{red}{\mathbf{u}^*}+R_{p}\right)\\
\mbox{Corrector :\quad}  D\textcolor{red}{\mathbf{u}^{n+1}}=D\textcolor{red}{\mathbf{u}^*}-U\textcolor{red}{p^{n+1}}\end{array}\right.
$$
Since in the Schnaps context the matrices $D$ and $M$ are exact we can compute exactly the Schur complement. Using this exact computation we obtain a preconditioning that call "Physic Based Exact" preconditioning.
Now the good strategy is not to construct exactly the Schur decomposition. \textcolor{red}{The other solution is to construct a discretization of the schur operator given by}
$$
\displaystyle \textcolor{red}{P_{schur}=(I_d-c^2\theta^2 \Delta t^2 \Delta )}
$$
This algorithm can be also obtain using a reformulation of wave equation and a splitting. Firstly we discretize in time the model
$$
\left\{\begin{array}{l}
\ds \frac{p^{n+1}-p^n}{\Delta t}+c\theta \nabla \cdot \bm{u}^{n+1}= f_p \\
\ds \frac{\bm{u}^{n+1}-\bm{u}^n}{\Delta t} +c\theta \nabla p^{n+1} = f_u \end{array}\right.
$$
Now we applying a splitting to obtain
$$
\left\{\begin{array}{l}
\ds \frac{\bm{u}^{*}-\bm{u}^n}{\Delta t}  = \mathbf{f}_u\\
~\\
\ds \frac{p^{n+1}-p^n}{\Delta t}+ c\theta \nabla \cdot  \mathbf{u}^{n+1}= f_p\\
~\\
\ds \frac{\bm{u}^{n+1}-\bm{u}^*}{\Delta t} +c\theta \nabla p^{n+1} = 0 \end{array}\right.
$$
Plugging the third equation in the second we obtain
$$
\left\{\begin{array}{l}
\ds \frac{\bm{u}^{*}-\bm{u}^n}{\Delta t}  = \mathbf{f}_u\\
~\\
\ds \frac{p^{n+1}-p^n}{\Delta t}- \Delta t \theta^2 c^2\Delta p^{n+1} = f_p-c\theta \nabla \cdot (\bm{u}^*) \\
~\\
\ds \frac{\bm{u}^{n+1}-\bm{u}^*}{\Delta t} +c\theta \nabla p^{n+1} = 0 \end{array}\right.
$$
Now the question is what this the good boundary condition to assure that the solution of the algorithm and the solution of the real problem are closed ( error in $\Delta t$). We want obtain at the boundary condition
$\lambda p^{n+1}- c (\bm{u}^{n+1},\bm{n})=g$. Using this $\bm{u}^{n+1}=-c \theta \Delta t\nabla p^{n+1}$ we obtain $\lambda p^{n+1}+ \theta \Delta t c^2 (\nabla p^{n+1},\bm{n})=g$. Consequently we obtain Robin boundary condition for $p$
$$
\nabla p\cdot \bm{n}=-\lambda\frac{p}{c(\Delta t\theta c)}+\frac{g}{c(\Delta t\theta c)}
$$
We will consider the implicit problem with $\mathbf{f}_u$ and $f_p$ generic RHS
$$
\left\{\begin{array}{l}
\ds \bm{u}^{*}= \mathbf{f}_u\\
~\\
\ds p^{n+1}- \Delta t c^2\theta^2\Delta p^{n+1} = f_p-c\theta \Delta t\nabla \cdot (\bm{u}^*) \\
~\\
\ds \bm{u}^{n+1}+c\Delta t\theta \nabla p^{n+1} = 0 \end{array}\right.
$$


At the end we introduce two test functions $v_p$ and $\mathbf{v}_u$. The weak form used in the preconditioning are given by
$$\left\{\begin{array}{l}
\ds \int_{\Omega} \bm{u}^{*} \mathbf{v}_u = \int_{\Omega}\mathbf{f}_u \mathbf{v}_u\\
~\\
\ds \int_{\Omega} p^{n+1} v_p+ \Delta t^2 \theta^2 c^2 \int_{\Omega}\nabla p^{n+1} \cdot \nabla v_p +\Delta t \theta\int_{\delta \Omega} \lambda p^{n+1} v_p\\
= \ds +c\theta\Delta t\int_{\Omega} \bm{u}^* \cdot \nabla v_p -c\theta\Delta t \int_{\delta \Omega} (\bm{u}\cdot\bm{n})v_p + \Delta t\int_{\Omega}f_p+\Delta t \theta \int_{\delta \Omega} \lambda g v_p\\
~\\
\ds \int_{\Omega} \bm{u}^{n+1} \mathbf{v}_u= c\theta\Delta t \int_{\Omega} p^{n+1} \nabla \cdot (\mathbf{v}_u)-c\theta\Delta t \int_{\delta \Omega} (\bm{v}_u\cdot\bm{n})p^{n+1}  \end{array}\right.
$$
We assume that the solution are regular. The first and third operators to invert are just the Identity consequently the two problem are well posed.The second one is a coercive problem with Robin Boundary condition. It is also known that this problem is well posed.

\subsection{Schur on the velocity}
To begin we write the same method but we compute the schur on the pressure. We obtain the following algorithm
The system is given by 
$$
\left(\begin{array}{ll} 
M  & U \\
 L & D\end{array}\right)\left(\begin{array}{l} 
  p^{n+1} \\
 \mathbf{u}^{n+1}
\end{array}\right)=\left(\begin{array}{l} 
 R_p\\
 R_{\mathbf{u}} \end{array}\right)
 $$
with $M=I_d$,
 $$
 D=\left(\begin{array}{ll} 
I_d  & 0 \\
 0 & I_d\end{array}\right),\quad
  U=\left(\begin{array}{ll} 
\ds\theta c\Delta t  \dx  & \ds\theta c\Delta t  \dy \end{array}\right),\quad
  L=\left(\begin{array}{l} 
\ds  \theta c \Delta t  \dx  \\
\ds \theta c \Delta t\dy \end{array}\right)
 $$
 and $\ds \mathbf{u}^{n+1}=\left(\begin{array}{ll} 
u_1^{n+1}  & u_2^{n+1} \end{array}\right)$. 
Consequently we can write the solution on the following form
$$
\left(\begin{array}{l} 
 p^{n+1} \\
  \mathbf{u}^{n+1}\end{array}\right) =\left(\begin{array}{ll}
M & U  \\
L & D\end{array}\right)^{-1}\left(\begin{array}{l} 
 R_p\\
 R_\mathbf{u} \end{array}\right)
$$
$$
=\left(\begin{array}{ll}
I & M^{-1} U  \\
0 & I\end{array}\right)\left(\begin{array}{ll}
M^{-1} & 0  \\
0 & P_{schur}^{-1}\end{array}\right)\left(\begin{array}{ll}
I & 0  \\
-LM^{-1} & I\end{array}\right)\left(\begin{array}{l} 
 R_\mathbf{u}\\
 R_p\end{array}\right)
 $$
with $\textcolor{red}{P_{schur}=D-LM^{-1}U}$. Solving the system with this decomposition is equivalent to solve the following algorithm 
$$
\left.\begin{array}{l}
\mbox{Predictor :\quad} M \textcolor{red}{p^*}=R_{p}\\
\mbox{Potential evolution :\quad} P_{schur}\textcolor{red}{\bm{u}^{n+1}}=\left(-L\textcolor{red}{p^*}+R_{\bm{u}}\right)\\
\mbox{Corrector :\quad}  M\textcolor{red}{^{n+1}}=M\textcolor{red}{p^*}-U\textcolor{red}{\bm{u}^{n+1}}\end{array}\right.
$$
Since in the Schnaps context the matrices $D$ and $M$ are exact we can compute exactly the Schur complement. Using this exact computation we obtain a preconditioning that call "Physic Based Exact" preconditioning.
Now the good strategy is not to construct exactly the Schur decomposition. \textcolor{red}{The other solution is to construct a discretization of the schur operator given by}
$$
\displaystyle \textcolor{red}{P_{schur}=(I_d-c^2\theta^2 \Delta t^2 \nabla \left( \nabla \cdot I_d \right))}
$$
This algorithm can be also obtain using a reformulation of wave equation and a splitting. Firstly we discretize in time the model
$$
\left\{\begin{array}{l}
\ds \frac{p^{n+1}-p^n}{\Delta t}+ c \nabla \cdot \bm{u}^{n+1}= f_p \\
\ds \frac{\bm{u}^{n+1}-\bm{u}^n}{\Delta t} +c\nabla p^{n+1} = f_u \end{array}\right.
$$
Now we applying a splitting to obtain
$$
\left\{\begin{array}{l}
\ds \frac{p^{n+1}-p^n}{\Delta t} = f_p\\
~\\
\ds \frac{\bm{u}^{n+1}-\bm{u}^*}{\Delta t} +c\nabla p^{n+1} = \mathbf{f}_u\\
~\\
 \ds \frac{p^{n+1}-p^n}{\Delta t}+ c\nabla \cdot  \mathbf{u}^{n+1}=0\end{array}\right.
$$
Plugging the third equation in the second we obtain
$$
\left\{\begin{array}{l}
\ds \frac{p^{*}-p^n}{\Delta t}  = f_p\\
~\\
\ds \frac{\bm{u}^{n+1}-\bm{u}^n}{\Delta t}- \Delta t^2 c^2 \nabla (\nabla \cdot \bm{u}^{n+1}) = \mathf{f}_u-c\nabla (p^*) \\
~\\
\ds \frac{p^{n+1}-p^n}{\Delta t}+ c\nabla \cdot  \mathbf{u}^{n+1}= 0 \end{array}\right.
$$
Now there are some questions
\begin{itemize}
\item What is the weak for the operator $I_d-\alpha \nabla \left( \nabla \cdot I_d \right))$.
\item What is the good boundary condition for the second step of the previous step
\item $I_d-\alpha \nabla \left( \nabla \cdot I_d \right))$ is invertible ? coercive ? well posed ? try with the classical weak form of the mixte weak form.
\item Well posed only for the irrotationnel velocity ?
\end{itemize}

\end{document}
