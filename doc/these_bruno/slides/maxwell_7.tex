\begin{frame}
\frametitle{Condition de saut}
La solution doit savoir prendre en compte les discontinuités fixes entre diélectriques : phénomène de réfraction.
\vfill
\scalebox{0.9}{\begin{minipage}{1.11\textwidth}
\begin{block}{Définition}
	Soit $\tilde{\n} = \Trp{(\Trp{\n},\nC{t})}$, avec $\n$
	un vecteur unitaire, la normale
	au support $\PbSrfTps$ d'une discontinuité spatio-temporelle de la solution du système
	de Friedrichs.
	La \textbf{relation de Rankine-Hugoniot} satisfaite sur cette discontinuité
	est donnée par l'équation :
	\begin{align*}
	\nC{t} \left( (\At)_\L \W_\L - (\At)_\R \W_\R \right) =
	\left( (\Aini)_\L \W_\L - (\Aini)_\R \W_\R \right) ,
	\end{align*}
	avec $\nC{t}$ la vitesse de propagation de la discontinuité.
\end{block}
\end{minipage}}
\vskip+1em
Application aux équations de Maxwell avec $\nC{t} = 0$ :
\begin{columns}
\column{0.5\textwidth}
\begin{align*}
\n \times (\H_\L - \H_\R) &= 0 , \\
\n \times (\E_\L - \E_\R) &= 0 .
\end{align*}
\column{0.5\textwidth}
\begin{align*}
\n \cdot (\EPrm_\L \E_\L - \EPrm_\R \E_\R) &= 0 , \\
\n \cdot (\HPrm_\L \H_\L - \HPrm_\R \H_\R) &= 0 .
\end{align*}
\end{columns}
\vfill
\begin{itemize}
\item Continuité des champs électromagnétiques tangents à $\PbSrfTps$ ;
\item Discontinuité des champs électromagnétiques normaux à $\PbSrfTps$.
\end{itemize}
\end{frame}

