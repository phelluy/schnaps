\begin{frame}
\frametitle{Problème semi-discret}
\vfill
Décomposition des champs électromagnétiques dans la base d'approximation :
\begin{align*}
	\W(\x,t) \approx \Wd(\x,t) &= \sum_{\L \in \Mesh} \sum_{i}
		\Wd_{\L,i}(t) \PsiPhy{\L}{i}(\x) ,
	\\
	\Ptl{t}\W(\x,t) \approx \Ptl{t}\Wd(\x,t) &= \sum_{\L \in \Mesh} \sum_{i}
		\Wd'_{\L,i}(t) \PsiPhy{\L}{i}(\x) ,
\end{align*}
avec $\Wd$ la solution du problème dit « semi-discret ».
\vfill
\begin{itemize}
\item Les seules valeurs aux points d'interpolation volumiques suffisent à définir la solution semi-discrète.
\item La solution du problème semi-discret $\Wd$
converge vers la solution du problème continu $\W$
lorsque le \textit{pas de maillage} tend vers zéro (théorème : Lesaint, Johnson).
\end{itemize}
\vfill
Pourquoi les hexaèdres ? Le choix de cet espace d'approximation implique des simplifications.
\vfill
\end{frame}
\renewcommand{\W}{\Vec{\mathrm{w}}} % Solution discrète dans la suite

