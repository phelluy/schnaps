\begin{frame}
\frametitle{Simplifications / Bilan}
\vfill
Pour $N_p$ fonctions de base (ou points d'interpolation), par maille :
\vfill
\begin{itemize}
\item Interpolation de Gauss-Legendre sur hexaèdre :
\begin{itemize}
\item Gradient du terme de volume : $O({N_p}^{4/3})$ opérations \textbf{scalaires} ;
\item Extrapolation du terme de flux : utilisation de $O(N_p)$ fonctions de base ;
\item Application du terme de flux : utilisation de $O(N_p)$ points de face.
\end{itemize}
\vfill
\item Interpolation sur tétraèdre :
\begin{itemize}
\item Gradient du terme de volume : $O({N_p}^{2})$ opérations \textbf{vectorielles} ;
\item Extrapolation du terme de flux : utilisation de $O({N_p}^{2})$ fonctions de base ;
\item Application du terme de flux : utilisation de $O({N_p}^{2})$ points de face.
\end{itemize}
\vfill
\item [=>] La complexité des calculs sur hexaèdre est inférieure à celle sur tétraèdre pour un ordre d'interpolation assez grand.
\end{itemize}
\vfill
Autre axe d'amélioration : l'intégration temporelle.
\vfill
\end{frame}

