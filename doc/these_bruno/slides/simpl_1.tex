\begin{frame}
\frametitle{Simplifications / Terme de masse}
\vfill
Pour une fonction de base $\PsiPhy{\L}{i}$ fixée :
\begin{align*}
	\int_{\L} \Ptl{t} \At \W \PsiPhy{\L}{i} d\x
	&\approx \int_{\L} \sum_{j}
		\At \W'_{\L,j} \PsiPhy{\L}{j} \PsiPhy{\L}{i} d\x
		\tag{Base} \\
	&= \int_{\HexaRef} \sum_{j}
		\At \W'_{\L,j}
		\PsiRef{j} \PsiRef{i}
		\Jac{\TGeo{\L}} d\xref
		\tag{Chgt. var.} \\
	&\approx \sum_{k} \sum_{j}
		\GLW{k} \At \W'_{\L,j}
		\PsiRef{j}(\GLN{k}) \PsiRef{i}(\GLN{k})
		\Jac{\TGeo{\L}}(\GLN{k})
		\tag{Quadrature} \\
	&= \GLW{i} \At \W'_{\L,i}
		\Jac{\TGeo{\L}} (\GLN{i})
		\tag{Termes nuls} .
\end{align*}
\vfill
\begin{itemize}
\item Le terme de masse fait intervenir une matrice diagonale.
\end{itemize}
\end{frame}

