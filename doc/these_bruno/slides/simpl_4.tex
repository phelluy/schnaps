\begin{frame}
\frametitle{Simplifications / Terme de flux (faces) conforme}
%\vfill
\vskip-1em
\begin{columns}%[c]
\column{.25\textwidth}
\begin{align*}
\QuadPhy_f \subset \Adh{\L},
\end{align*}
\column{.25\textwidth}
\begin{align*}
\QuadPhy_{f'} \subset \Adh{\R},
\end{align*}
\column{.25\textwidth}
\begin{align*}
\QuadPhy_f = \QuadPhy_{f'},
\end{align*}
\column{.25\textwidth}
\begin{align*}
\Deg_\L = \Deg_\R = \Deg,
\end{align*}
\end{columns}
\vskip+0.5em
Il existe une permutation d'indices qui fait correspondre les points de face :
\vskip-2em
\begin{align*}
	\sigma : \EnsN \rightarrow \EnsN :
	\GLNPhy{\L}{f,i} = \GLNPhy{\R}{f',\sigma(i)} , \
	\TGeo{\L}(\GLN{f,i}) = \TGeo{\R}(\GLN{f',\sigma(i)}) ,
\end{align*}
\vfill
Pour une fonction de base $\PsiPhy{\L}{i}$ fixée :
\scalebox{0.7}{\begin{minipage}{1.42\textwidth}
\begin{align*}
	&\int_{\QuadPhy_f} \Flux{\W_\L}{\W_\R}{\n} \PsiPhy{\L}{i} ds \\
	&= \int_{\QuadRef_f}
		\Flux{\W_\L \circ \TGeo{\L}}{\W_\R \circ \TGeo{\L}}{
			\Com{\TGeo{\L}'} \hat{\n}}
		\PsiPhy{\L}{i} \circ \TGeo{\L} d\hat{s}
		\tag{Chgt. var.} \\
	&\approx \int_{\QuadRef_f}
		\Flux{\sum_{k}
		\W_{\L,k} \PsiRef{k}}{\sum_{k}
		\W_{\R,k} \PsiRef{k} \circ \Inv{\TGeo{\R}} \circ \TGeo{\L}}{
			\Com{\TGeo{\L}'} \hat{\n}}
		\PsiRef{i} d\hat{s}
		\tag{Base} \\
	&\approx \sum_{j} \GLW{f,j}
		\Flux{\sum_{k}
		\W_{\L,k} \PsiRef{k}(\GLN{f,j})}{\sum_{k}
		\W_{\R,k} \PsiRef{k}(\GLN{f',\sigma(j)})}{
			\Com{\TGeo{\L}'} \hat{\n}}
		\PsiRef{i}(\GLN{f,j})
		\tag{Quadrature} \\
	&= \GLW{f,\pi(f,i)}
		\Flux{\sum_{k}
		\W_{\L,k} \PsiRef{k}(\GLN{f,\pi(f,i)})}{\sum_{k}
		\W_{\R,k} \PsiRef{k}(\GLN{f',\sigma(\pi(f,i))})}{
			\Com{\TGeo{\L}'} \hat{\n}}
		\PsiRef{i}(\GLN{f,\pi(f,i)})
		\tag{Termes nuls}
\end{align*}
\end{minipage}}
\vfill
\vfill
\begin{itemize}
\item Flux : extrapolation des champs sur les points de face ;
\item Application des flux issus uniquement des projetés orthogonaux ;
\item Des simplifications apparaissent à l'extrapolation des champs sur la face.
\end{itemize}
\vfill
\end{frame}

