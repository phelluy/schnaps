\vspace*{\fill}
\begin{flushright}
\parbox{0.3\linewidth}{
  \textit{Je dédie cette thèse à Georgette,
    ma mère, et Gérard, mon père,
    qui m'ont permis d'arriver jusque là.}
}
\\
\parbox{0.3\linewidth}{
 \begin{flushright}
 \textit{Merci.}
 \end{flushright}
}
\end{flushright}
\vspace*{\fill}

\chapter*{Remerciements}


Je tiens tout d'abord à remercier
Philippe Helluy, directeur de cette thèse,
qui a toujours été présent pour répondre à mes questions,
qui m'a soutenu au cours de ces derniers longs mois de rédaction
et sans qui je n'aurais pas eu l'opportunité de réaliser
ces intéressants travaux de recherche et d'optimisations.

Dans un second temps, je remercie
Christophe Girard, PDG de la société AxesSim,
qui a permis à cette thèse de voir le jour au format CIFRE.
Je reconnais par la même ses qualités telles que la souplesse dont
il fait preuve dans la gestion de l'entreprise ainsi que son
engagement dans le développement de la méthode Galerkin Discontinue (GD).

Mes remerciements s'adressent aussi à Hélène Barucq et Xavier Ferrières
qui ont accepté de rapporter cette thèse.
Je remercie également Christophe Prud'homme
d'avoir accepté de faire partie du jury en tant qu'examinateur.

Le bon déroulement de cette thèse est aussi dû aux bons conseils,
à l'expérience et l'aide apportés par toute l'équipe AxesSim.
Je remercie notamment Cyril Giraudon pour le partage de ses vastes
connaissances,
Thomas Strub pour m'avoir épaulé à mes débuts sur le code de calcul GD,
Nathanaël Muot pour sa vigilance quant au respect des délais,
Didier Roissé pour sa passion de la compilation,
Philippe Spinosa pour sa patience envers les \textit{noobs},
Guillaume Prin et Xavier Romeuf pour leur réactivité
et Sophie Marin pour son efficacité et sa bonne humeur.

Du côté universitaire, je remercie particulièrement
Malcolm Roberts avec qui j'ai assisté à mes premières conférences
ainsi que tout le personnel administratif, autant de l'IRMA que
de l'école doctorale, pour m'avoir facilité les démarches administratives.

Merci aussi à tous mes proches, famille, amis, amie, de m'avoir
accompagné tout au long de cette thèse (pour certains bien avant déjà),
de s'être intéressé à mes travaux et de m'avoir écouté
(me plaindre, parfois).

Enfin, je vous remercie par avance, vous, lecteurs, d'avoir ouvert
cette thèse et vous souhaite une bonne lecture.
