\chapter*{Introduction}
\addcontentsline{toc}{chapter}{Introduction}


Cette thèse a été réalisée dans le cadre du projet HOROCH\footnote{HOROCH : utilisation des HPC pour l'Optimisation des Radiocommunications
des Objets Connectés proches de l'Homme}
au sein de l'entreprise AxesSim, en collaboration avec
Thales Communications \& Security, les centres de recherche
IRMA Strasbourg et ONERA Toulouse, les PME Cityzen Sciences et BodyCap,
et financé par la DGE\footnote{Direction Générale des Entreprises}.

L'objectif de ce projet était de développer, en parallèle,
des solutions de simulation et de mesure du rayonnement
électromagnétique des objets connectés placés à proximité
(ou à l'intérieur) du corps humain, afin de fournir des outils avancés
de conception d'objets connectés (IoT, pour « \textit{Internet of Things} »)
aux PME de ce secteur d'activité.
Les domaines d'application sont nombreux : téléphones mobiles,
montres connectées, maillots connectés, sondes de mesure médicales
(gélules ou patchs), \textit{etc}.

Ainsi, TCS avait en charge la conception d'un banc de mesure
sous la forme d'une arche composée de $32$ récepteurs Vivaldi.
AxesSim avait en charge l'évolution de son solveur
Galerkin Discontinu (GD) \texttt{teta}.
L'IRMA, en collaboration avec AxesSim au travers de thèses CIFRE,
est en charge du solveur GD générique \texttt{clac}
(pour « \textit{Conservation Laws on mAny Cores} »),
brique de base du solveur \texttt{teta}, spécialisé pour
des applications en l'électromagnétisme.
\\


Pour AxesSim, l'objectif final de ce projet,
et par extension de cette thèse, était d'effectuer une simulation
d'une antenne BLE (pour « \textit{Bluetooth Low Energy} ») placée
à proximité d'un modèle de corps humain complet (avec squelette et organes). Ce modèle a été obtenu par scan à rayons X,
effectué par l'IRCAD de Strasbourg d'un mannequin anthropomorphique
approvisionné par TCS.

Le maillage non structuré du mannequin complet comporte
environ $24$ millions de mailles hexaédriques
pour des applications entre $1$ et $3$ GHz, bande de fréquence des IoT.
Les antennes utilisées ont généralement une taille de l'ordre de grandeur
de la longueur d'onde du signal émis (soit $1$-$10$ cm dans notre cas),
avec des adaptations ou des composants d'un voire deux ordres de grandeur
inférieurs en taille.

Dans le cas de cette étude, le facteur d'échelle entre la plus petite
maille (antenne) et la plus grande maille (vide englobant) est de l'ordre
de $1000$. Ces variations dans la taille des mailles en fait un cas
d'application bien adapté à la méthode GD, contrairement à la 
méthode des différences finies, plus fréquemment utilisée, qui nécessiterait
de mailler toute la scène de manière structurée tout en respectant
au mieux la géométrie des plus petits détails de l'antenne.

La première simulation sur corps humain complet a été effectuée en $11$ jours
sur une machine de calcul composée de $8$ GPU NVidia de dernière génération
pour seulement $10$ ns de temps physique.
Cette simulation a été exécutée avec un faible ordre d'interpolation
et a nécessité plus de $40$ Go de mémoire vive.
Afin de permettre au solveur \texttt{teta-clac} de traiter des simulations
de cette taille, un grand nombre d'optimisations et
d'améliorations algorithmiques ont été nécessaires.
Nous allons présenter les travaux qui ont été menés afin d'atteindre
ces performances.
\\



Dans le chapitre \ref{chap:generalites}, nous commençons
par rappeler la formulation des équations de Maxwell.
Ce système de $4$ équations, auxquelles nous ajoutons
l'équation de conservation de la charge, régit
la propagation des ondes électromagnétiques et nous permettra
de les modéliser par la simulation numérique.

Nous écrivons ensuite ces équations sous la forme générique d'un
système hyperbolique symétrique linéaire du premier ordre,
aussi appelé système de Friedrichs. De nombreuses propriétés
sont connues pour ces systèmes d'équations aux dérivées partielles. Nous examinons
ensuite les possibles discontinuités des solutions de ce système
et donnons les conditions d'existence et d'unicité d'une
telle solution.

Enfin, nous introduisons le schéma GD
qui permet d'approcher le système hyperbolique. Le schéma GD est caractérisé par une formulation faible avec
des fonctions test discontinues et un flux numérique sur les discontinuités.
Nous listons alors quelques flux numériques permettant
d'assurer la stabilité du schéma.
Nous donnons aussi une formulation semi-discrète du problème
qui a servi à l'implémentation informatique du solveur
 générique \texttt{clac}.
Des propriétés de convergence mathématique du problème semi-discret
sont rappelées en annexe \ref{annexe:demo_cv}.
\\



Dans le chapitre \ref{chap:modeles}, nous présentons les
principaux modèles physiques nécessaires à la résolution
des problèmes qui nous intéressent.
Ces modèles sont implémentés dans la surcouche applicative
\texttt{teta} développée par AxesSim, spécialisation à l'électromagnétisme
du solveur générique \texttt{clac}.

Nous commençons par décrire des modèles de conditions
de bord respectant les conditions d'unicité de la
solution.
Nous donnons aussi le détail de la construction
du modèle de couches absorbantes PML
(pour « \textit{Perfectly Matched Layer} »)
permettant de résoudre un problème en domaine infini simulé.

Nous présentons ensuite des techniques numériques pour représenter
de fines plaques de matériaux.
La suppression géométrique de ces plaques minces
permet de simplifier la modélisation de la scène
étudiée et diminue les temps de calcul du solveur.
L'un de ces modèles de plaque mince sera également étendu en un
modèle permettant l'injection de tensions ou de courants
électriques utilisés dans la simulation de dispositifs rayonnants.

Enfin, dans le but de traiter les matériaux diélectriques dont la
permittivité varie en fonction de la
fréquence,
nous décrivons aussi un modèle volumique prenant en compte ces
variations.
Ce type de modèle est intéressant dans le cas de simulations
faisant intervenir le corps humain qui présente ce type de matériaux.
\\



Dans le chapitre \ref{chap:implementation}, nous décrivons
l'implémentation générique du solveur \texttt{clac}
programmé pour traiter les mailles hexaédriques à arêtes droites.
La méthode GD peut théoriquement être appliquée à
tout type de maillage, notamment sur mailles tétraédriques.
Les hexaèdres permettent diverses optimisations de la formulation GD et
sont bien adaptés à la parallélisation sur processeur graphique.

Nous commençons par introduire l'espace d'approximation
dans lequel nous chercherons la solution numérique de notre problème.
Cet espace est engendré par des fonctions de base à support
inclus dans une unique maille.
Dans le cas du solveur \texttt{teta-clac} nous utilisons
une base de polynômes constituée
des produits tensoriels des polynômes d'interpolation de Lagrange,
eux-mêmes générés à partir des points d'intégration de Gauss-Legendre.

Nous présentons aussi les schémas d'intégration en temps
actuellement les plus souvent utilisés par le solveur :
les schémas de type Runge-Kutta (RK).
Ces schémas explicites imposent une condition
 de type CFL afin d'assurer
la stabilité du schéma couplé espace-temps (le plus
souvent GD-RK$2$).
Nous donnons la formulation des $2$
conditions CFL utilisées pour les cas d'application
du dernier chapitre ainsi que $2$ diagnostics de stabilité.
\\




Dans le chapitre \ref{chap:parallelisations_et_optimisations},
nous décrivons l'implémentation parallèle sur (un ou) plusieurs
accélérateurs de la méthode GD discrète décrite au chapitre précédent.
Pour cela, nous utilisons la bibliothèque OpenCL
(pour « \textit{Open Computing Language} ») qui permet
d'exploiter la puissance de calcul de ces accélérateurs.
Les cartes graphiques (GPU) récentes permettent de réaliser des calculs
de façon massivement parallèle, et sont de ce fait
très utilisées dans la simulation numérique.
L'implémentation GPU a initialement été réalisée en C++ en utilisant la bibliothèque
OpenCL au cours de la thèse de T. Strub \cite{strub:tel-01651258}, à partir du code non parallèle
FemGD développé à l'ONERA de Toulouse.

Afin de pouvoir traiter des volumes de données importants, la méthode GD est
également parallélisée sur plusieurs accélérateurs à l'aide d'une implémentation du standard
MPI (« \textit{Message Passing Interface} »).
Dans notre cas, MPI offre la possibilité de lancer plusieurs
processus qui effectuent les calculs sur différentes parties du maillage
et communiquent par un système de messages.

Cette implémentation GPU est aussi exploitable sur un processeur multi-cœurs
(CPU) plus classique. Ces derniers ont beaucoup évolué au cours des années passées
pour être aujourd'hui composés de plusieurs dizaines de cœurs logiques cadencés
à des fréquences supérieures à celles des GPU. Cependant, de par leur architecture
radicalement différente de celle des GPU, une implémentation spécifique aux CPU doit être
développée afin de les exploiter au mieux.

Nous présentons aussi une méthode d'adaptation de l'ordre d'interpolation
spatiale appliquée aux mailles du maillage dans le but de réduire le temps
de simulation en optimisant l'échantillonnage du domaine de calcul.
\\




Dans le chapitre \ref{chap:pas de temps local}, nous présentons une
adaptation algorithmique permettant d'améliorer encore la vitesse de calcul
du solveur : un schéma à pas de temps local par maille.
Ce schéma permet de réduire la quantité de calculs et ne nécessite
donc pas d'accélérateur supplémentaire pour obtenir de meilleures
performances.
Cette méthode d'intégration est théoriquement très efficace
mais n'est \textit{a priori} adaptée qu'aux maillages présentant un facteur
d'échelle important entre la plus petite et les plus grandes mailles.

Nous commençons par introduire une variante locale du schéma
RK d'ordre $2$ (RK$2$) que nous appelons LRK$2$ (pour « \textit{Local} RK$2$ »).
Nous utilisons ensuite les propriétés locales du schéma LRK$2$ afin
de découpler l'avancée en temps de l'approximation de la solution sur
les différentes mailles. Ainsi, ce second schéma, que nous appelons
LTS$2$ (pour « \textit{Local Time Step} $2$ »), permet de réduire
considérablement le temps de simulation en divisant la quantité
de calculs lorsque la géométrie est composée de mailles de tailles très hétérogènes.
\\



Dans le chapitre \ref{chap:runtimes}, nous montrons dans quelle
mesure un solveur GD s'adapte à l'exécution sur ordinateur hybride.
Les (super)calculateurs actuels sont généralement un regroupement
de nœuds de calcul identiques, composés d'un (ou plusieurs) CPU et éventuellement d'un GPU.
L'équilibrage de la charge de calcul est donc aisée
entre périphériques de même nature.
Cependant, les CPU sont très différents
des GPU, tant au niveau de l'architecture que de la puissance de calcul.
Ainsi, afin d'être capable d'utiliser toute la puissance de calcul
d'un nœud hybride, il est très intéressant
de savoir équilibrer la charge ce calcul entre un CPU et un GPU.

A ces fins, nous avons utilisé la bibliothèque StarPU
développée à l'INRIA de Bordeaux.
Cette bibliothèque permet de se défaire de l'équilibrage manuel
des tâches via un graphe des tâches. Ce dernier est automatiquement
généré par StarPU qui utilise la dépendance des données
passées en entrée et en sortie des tâches.
Les tâches sont alors soumises aux périphériques automatiquement
et les données sont transférées en fonction du besoin.

Les expérimentations de l'utilisation de cette bibliothèque ont été
effectuées sur le solveur universitaire \texttt{schnaps}
et sont présentées sous la forme
d'un article qui sera prochainement soumis.
\\




Enfin, dans le chapitre \ref{chap:validation}, nous donnons
un ensemble de résultats validant notre implémentation de solveur GD,
autant par la précision des calculs que par leur vitesse d'exécution.

Nous commençons par présenter des résultats de convergence
sur un cas académique de propagation d'une onde plane dans un cube.
Nous en profitons pour effectuer des comparaisons entre maillages
structurés (hexaèdres droits) et non structurés (issus du découpage de
tétraèdres).

Nous comparons ensuite notre solveur GD à un solveur implémentant
la méthode des différences finies sur un cas d'application
mettant en scène une antenne dipolaire à proximité de l'oreille d'une tête
humaine simplifiée.
Ce cas d'application est un modèle simplifié de rayonnement d'un téléphone portable.
Nous donnons aussi les résultats, sur ce même cas d'application,
des différents schémas d'intégration en temps que nous avons évoqués plus tôt.


Nous présentons ensuite des calculs réalisés sur un modèle de corps humain complet, à côté duquel
rayonne une antenne BLE.
Il s'agit de la plus grande simulation réalisée par le solveur à ce jour.
Ce calcul a été réalisé sur un calculateur monté par AxesSim et disposant de 8 cartes graphiques.

Enfin, dans le cadre d'un appel à projets PRACE, nous avons pu avoir accès au
supercalculateur PizDaint. Cela nous a permis, sur le cas d'application du corps humain complet, de réaliser un
test de scalabilité forte du code GD sur plusieurs centaines de GPU.
