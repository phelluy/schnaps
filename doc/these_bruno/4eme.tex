\documentclass[a4paper,12pt]{report}
\usepackage[hmargin=1.5cm,vmargin=2cm]{geometry}
\usepackage[T1]{fontenc}
\usepackage[utf8]{inputenc}
\usepackage{graphicx}
\graphicspath{{img/}} % Répertoire des images

\title{\huge{\textbf{Optimisation de code Galerkin Discontinu sur ordinateur hybride. Application à la simulation numérique en électromagnétisme.}}}

\author{Bruno Weber}

\begin{document}

\begin{titlepage}
	\centering
  $
  \begin{array}{l}
	\includegraphics[width=0.18\linewidth]{irma}
  \end{array}
  $
	\hfill
	\Large\textbf{Bruno WEBER}
	\hfill
  $
  \begin{array}{l}
	\includegraphics[width=0.18\linewidth]{axs}
  \end{array}
  $
	\vfill
	\Large\textbf{Optimisation de code Galerkin Discontinu
		sur ordinateur hybride. Application à la simulation
		numérique en électromagnétisme.}\\
	\vfill
	\framebox{
		\parbox{0.9\linewidth}{
\normalsize
\textbf{Résumé :}\\
Nous présentons dans cette thèse les évolutions apportées au
solveur Galerkin Discontinu \texttt{teta-clac},
issu de la collaboration IRMA-AxesSim,
au cours du projet HOROCH (2015-2018).
Ce solveur permet de résoudre les équations de Maxwell en
3D, en parallèle sur un grand nombre d'accélérateurs OpenCL.

L'objectif du projet HOROCH était d'effectuer des simulations
de grande envergure sur un modèle numérique complet de corps humain.
Ce modèle comporte 24 millions de mailles
hexaédriques pour des calculs dans la bande de fréquences
des objets connectés allant de 1 à 3 GHz (Bluetooth).
Les applications sont nombreuses : téléphonie et accessoires,
sport (maillots connectés), médecine (sondes : gélules, patchs),
\textit{etc}.

Les évolutions ainsi apportées comprennent, entre autres :
l'optimisation des kernels OpenCL à destination des CPU
dans le but d'utiliser au mieux les architectures hybrides~;
l'expérimentation du runtime StarPU~; le design d'un
schéma d'intégration à pas de temps local~;
et bon nombre d'optimisations permettant au solveur
de traiter des simulations de plusieurs millions de mailles.

~\\
\textbf{Mots-clés :}\\
\raggedright
solveur, Maxwell, électromagnétisme, système hyperbolique,
Galerkin Discontinu, GD, GDTD, maillage, hexaèdres,
GPU, CPU, OpenCL, MPI, StarPU,
pas de temps local, ordre adaptatif, modèle de corps humain complet,
objets connectés, Bluetooth.
		}
	}
	\vfill
	\framebox{
		\parbox{0.9\linewidth}{
\normalsize
\textbf{Abstract:}\\
In this thesis, we present the evolutions made to the
Discontinuous Galerkin solver \texttt{teta-clac} --- resulting from
the IRMA-AxesSim collaboration --- during the HOROCH project (2015-2018).
This solver allows to solve the Maxwell equations in 3D and in parallel
on a large amount of OpenCL accelerators.

The goal of the HOROCH project was to perform large-scale simulations
on a complete digital human body model.
This model is composed of 24 million hexahedral cells in order
to perform calculations in the frequency band of connected objects
going from 1 to 3 GHz (Bluetooth).
The applications are numerous: telephony and accessories, sport
(connected shirts), medicine (probes: capsules, patches), \textit{etc}.

The changes thus made include, among others: optimization of OpenCL kernels
for CPUs in order to make the best use of hybrid architectures;
StarPU runtime experimentation; the design of an integration scheme using
local time steps; and many optimizations allowing the solver to process
simulations of several millions of cells.

~\\
\textbf{Keywords:}\\
\raggedright
solver, Maxwell, electromagnetism, hyperbolic system,
Discontinuous Galerkin, DG, DGTD, mesh, hexahedrons,
GPU, CPU, OpenCL, MPI, StarPU,
local time step, adaptive order, complete human body model,
connected objects, Bluetooth.
		}
	}
\end{titlepage}

\end{document}
