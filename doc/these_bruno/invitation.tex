\documentclass[a4paper,12pt]{report}
\usepackage[hmargin=1.5cm,vmargin=2cm]{geometry}
\usepackage[T1]{fontenc}
\usepackage[utf8]{inputenc}
\usepackage{graphicx}
\graphicspath{{img/}} % Répertoire des images

\title{\huge{\textbf{Optimisation de code Galerkin Discontinu sur ordinateur hybride. Application à la simulation numérique en électromagnétisme.}}}

\author{Bruno Weber}

\begin{document}

\begin{titlepage}
  \centering
  $
  \begin{array}{l}
    \includegraphics[width=0.18\linewidth]{uds}
  \end{array}
  $
  \hfill
  \Large\textbf{INVITATION}
  \hfill
  $
  \begin{array}{l}
    \includegraphics[width=0.18\linewidth]{axs}
  \end{array}
  $
  \vfill
  \Large\textbf{Optimisation de code Galerkin Discontinu
    sur ordinateur hybride. Application à la simulation
    numérique en électromagnétisme.}\\
  \vfill
  \Large\textbf{Soutenance de thèse de Bruno WEBER}
  \vfill
  \large\textbf{Lundi 26 novembre 2018 à 11h00}
  \vfill
  \large\textbf{Salle de conférences\\
    Institut de Recherche Mathématique Avancée (IRMA)\\
    7 Rue René Descartes\\
    67000 Strasbourg
  }
  \vfill

  \parbox{0.9\linewidth}{
    \normalsize
    \textbf{Jury :}\\
    \begin{tabular}{lllll}
      \textbf{M Helluy Philippe} & &
      Université de Strasbourg & & Directeur de thèse \\
      \textbf{Mme Barucq Hélène} & &
      INRIA Bordeaux & & Rapporteur \\
      \textbf{M Ferrières Xavier} & &
      ONERA Toulouse & & Rapporteur \\
      \textbf{M Prud'homme Christophe} & &
      Université de Strasbourg & & Examinateur \\
      \textbf{M Giraudon Cyril} & &
      AxesSim, Illkirch-Graff. & & Encadrant industriel \\
    \end{tabular}
  }
  \vfill
  \parbox{0.9\linewidth}{
    \normalsize
    \textbf{Résumé :}\\
Nous présentons dans cette thèse les évolutions apportées au
solveur Galerkin Discontinu \texttt{teta-clac},
issu de la collaboration IRMA-AxesSim,
au cours du projet HOROCH (2015-2018).
Ce solveur permet de résoudre les équations de Maxwell en
3D, en parallèle sur un grand nombre d'accélérateurs OpenCL.

L'objectif du projet HOROCH était d'effectuer des simulations
de grande envergure sur un modèle numérique complet de corps humain.
Ce modèle comporte 24 millions de mailles
hexaédriques pour des calculs dans la bande de fréquences
des objets connectés allant de 1 à 3 GHz (Bluetooth).
Les applications sont nombreuses : téléphonie et accessoires,
sport (maillots connectés), médecine (sondes : gélules, patchs),
\textit{etc}.

Les évolutions ainsi apportées comprennent, entre autres :
l'optimisation des kernels OpenCL à destination des CPU
dans le but d'utiliser au mieux les architectures hybrides~;
l'expérimentation du runtime StarPU~; le design d'un
schéma d'intégration à pas de temps local~;
et bon nombre d'optimisations permettant au solveur
de traiter des simulations de plusieurs millions de mailles.

~\\
\textbf{Mots-clés :}\\
\raggedright
solveur, Maxwell, électromagnétisme, système hyperbolique,
Galerkin Discontinu, GD, GDTD, maillage, hexaèdres,
GPU, CPU, OpenCL, MPI, StarPU,
pas de temps local, ordre adaptatif, modèle de corps humain complet,
objets connectés, Bluetooth.
  }
\end{titlepage}

\end{document}
