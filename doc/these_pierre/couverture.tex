%\documentclass{Couv}
% REMARQUES :
% compiler ce fichier avec le commande pdflatex.
% il peut etre necessaire de compiler deux fois le fichier.


% Si besoin ajouter ici 
% - les packages necessaires
% - les (re)definition de commandes TeX
\def \T{{\bf T}}
\def \R{{\bf R}}
\def \Z{{\bf Z}}
%\begin{document}
% Prenom et Nom en miniscules
\author{Pierre GERHARD}

% Date au format suivant : jour (un ou deux chiffres) mois (en toutes lettres) et annee (sur quatre chiffres)
\date{XX mois 201X}

% Titre en minuscules
\title{Réduction de modèles cinétiques et applications à l'acoustique du bâtiment} 

% Liste des membres du jury (pour les theses et HDR uniquement)
% Pour chaque membre : Prenom Nom (en minuscule) et fonction (rapporteur, examinateurs,... en minuscule aussi) 
% Inserer \\ entre chaque membre pour les seprarer d'un retour a la ligne
\jury{
Philippe Helluy, directeur de th\`ese \\
Cédric Foy, co-encadrant\\
Laurent Navoret, co-encadrant\\
Prénom NOM, rapporteur \\
Prénom NOM, rapporteur \\
Prénom NOM, examinatrice
}

% Resume en minuscules
\resume{Résumé.}



% Numero d'ordre atribue par le secretariat de l'IRMA (de la forme annee/numero)
\issnordre{20XX/0X}

% Adresse de depor sur le serveur TEL
\telnum{http://tel.archives-ouvertes.fr/tel-XXXXXXXX}


% commandes disponibles pour creer la premiere de couverture (decommenter la ligne necessaire) :
 % \makethesemath % pour une these, specialite MATHEMATIQUES
\makethesemathapp % pour une these, specialite MATHEMATIQUES APPLIQUEES
% \makehdrmath % pour une HDR, specialite MATHEMATIQUES
% \makehdrmathapp % pour une HDR, specialite MATHEMATIQUES APPLIQUEES
% \makeprepubli % pour une prepublication

% commande disponible pour creer la quatrieme de couverture :
\makeresume
%\end{document}